\documentclass[a4paper,12pt]{article} 
% Paquetes......................................................................
\usepackage{amsmath, amssymb, amsfonts, latexsym}
\usepackage[utf8]{inputenc}
\usepackage[T1]{fontenc}
\usepackage{palatino}
\usepackage[full]{textcomp}
\usepackage{hyperref}
\usepackage{eurosym}
\usepackage[makeroom]{cancel}
\usepackage{array}
\usepackage{pdfpages}
\usepackage{float} % para que las figuras no floten
\usepackage{subcaption}

\textheight = 24 cm
\textwidth = 17 cm

\renewcommand{\arraystretch}{1.25}
\renewcommand{\contentsname}{Contenidos}


% INICIO DEL DOCUMENTO --------------------------------------------------------
\begin{document}
	
	\setlength{\parindent}{0.5cm}
	\setlength{\voffset}{-2cm}
	\setlength{\hoffset}{-2cm}
	
	\input{./include/portada.tex}
	
	\tableofcontents
	
\newpage

	\section{Introducción}
	% Introduction. High level overview of the ontology you plan to develop and brief description of the datasets and web pages that you plan to use.
	
	\section{Metodología NeOn}
	% Make a short overview of the NeOn methodology making explicit:
	%   a. which scenarios you plan to follow
	%   b. which activities from the glossary of activities you plan to execute in your ontology development project
	
	\section{Especificación de la ontología}
	% Ontology specification. You should include a complete ontology specification requirement document using the template explained during the lectures. The goal of the ontology should be clear enough. The document should include relevant competency questions and their answers.
	
	\section{Planificación temporal de la ontología}
	% Ontology Schedule. You should identify the ontology life cycle model you use in your ontology development. Activities should come from the glossary of terms. Include a Gantt chart with the planned activities.
	
	\section{Búsqueda de ontologías de alto nivel}
	% Search for existing top level and domain ontologies that you plan to reuse when building your ontology. Explain if you need to reuse the ontology as a whole or if you need some modules or statements. Explain in which repositories you made the search, ontologies found, their relevance to your work and the criteria being used for selecting or withdrawing some of them.
	
	\section{Recursos no ontológicos}
	%Search for non ontological resources and other terminologies that could be transformed into ontologies. Keep track of the URLs where you found them. For the selected resources, do not forget to justify why you have selected them. Check if you have Access rights for using them within your hands-on assigment.
	
	\section{Patrones en la ontología}
	% Search for some ontology design patterns in the ontology design pattern portal that could be reused in your development.
	
	\section{Modelo conceptual}
	% Build a conceptual model that integrates outcomes from the previous sections (c,d,e). This is the most important part of the work you are doing. Try to use:
	%  	i. Top level ontologies and other well-known ontologies. Classify in the pyramid of ontologies (figure use vs reuse) each of the ontologies that you reuse.
	%   ii. Transform each non-ontology resource into an ontology by using the T-box, A-Box or Population.
	%   iii. Select some Ontology Design Patterns (events, sequence, etc.)
	%   iv. Build the conceptual model of your ontology by integrating the above sources. The conceptual model should have at least 40 concepts, several subclass-of relations, disjoint, part-of (if needed), and ad-hoc relations.
	
	\section{Clases Multilingües}
	
	\section{Implementación de la ontología con OWL}
	% Implement the ontology in an ontology development tool, or other ontology editor, using OWL as ontology language
	
	\section{Evaluación de la ontología con OOPS!}
	% Evaluate the ontology with OOPS! and include in your report the pitfalls found. It is highly advisable to combine this evaluation with other ontology evaluation techniques.
	
	\section{Evaluación. Mejoras propuestas}
	% Improve your ontology (conceptual model and implementation) taking into account the suggestions given by OOPS!. Iterate in these steps until the ontology pass most of the OOPS! recommedations.
	
	\section{Documentación de la ontología}
	% Document the ontology with Widoco and use Ontoology if required
	
	\section{Conclusiones}
	
	
\newpage
	\section*{Bibliografía}
	\addcontentsline{toc}{section}{Bibliografía}
	\bibliography{include/references}
	\bibliographystyle{IEEEtran}

\end{document}