\documentclass[a4paper,12pt]{article} 
% Paquetes......................................................................
\usepackage{amsmath, amssymb, amsfonts, latexsym}
\usepackage[utf8]{inputenc}
\usepackage[T1]{fontenc}
\usepackage{palatino}
\usepackage[full]{textcomp}
\usepackage{hyperref}
\usepackage{eurosym}
\usepackage[makeroom]{cancel}
\usepackage{array}
\usepackage{pdfpages}
\usepackage{float} % para que las figuras no floten
\usepackage{subcaption}
\usepackage{soul} % para el highlight \hl
\usepackage{pdfpages} % para pegar otro pdf dentro de este

\textheight = 24 cm
\textwidth = 17 cm

\renewcommand{\arraystretch}{1.25}
\renewcommand{\contentsname}{Contenidos}


% INICIO DEL DOCUMENTO --------------------------------------------------------
\begin{document}
	
	\setlength{\parindent}{0.5cm}
	\setlength{\voffset}{-2cm}
	\setlength{\hoffset}{-2cm}
	
	\input{./include/portada.tex}
	
	\tableofcontents
	
\newpage

	\section{Introducción}
	% Introduction. High level overview of the ontology you plan to develop and brief description of the datasets and web pages that you plan to use.
		
	En el contexto de la iniciativa OpenCityData \cite{opencitydata} y del proyecto
Ciudades Abiertas \cite{ciudadesabiertas} se han desarrollado un buen número de
ontologías que permitan representar datos abiertos de ciudades, en muy diversas áreas (transporte,
gestión económica, equipamientos, servicios, etc.). Dentro de esta iniciativa se encuentran
muchas áreas todavía por desarrollar, y es una de ellas en las que se enmarca el desarrollo de
nuestro trabajo.
	
	El objetivo es diseñar e implementar una ontología que represente de manera adecuada
instalaciones deportivas, sus características y acciones relacionadas. Las ontologías y otras
fuentes de conocimiento utilizadas durante el desarrollo de este trabajo serán citadas, y se puede
acceder a ellas a través de los hipervínculos de la bibliografía.
La ontología por desarrollar constituirá un modelo para representar los datos asociados a una
instalación deportiva, donde se entiende como instalación deportiva un recinto o construcción provista de los medios necesarios para la práctica, aprendizaje y competición de uno o más
deportes, y que está asociada, organizada y mantenida por una determinada organización. Como
iremos describiendo en los siguientes apartados de la memoria, se especificarán todos los
requisitos necesarios para la creación de la ontología, y se detallarán todas las características de
esta.
	
	A la hora de la creación de la ontología se han seguido las pautas que marca la metodología NeOn,
haciendo uso tanto de recursos ontológicos como no ontológicos para su reutilización y
reingeniería. Cada una de las actividades de la metodología han sido realizadas en una secuencia
especificada y con un planteamiento ordenado, y tras ello se ha usado la herramienta  para
la formalización en OWL de la ontología y OOPS! para su evaluación.
	
	\section{Metodología NeOn}
	% Make a short overview of the NeOn methodology making explicit:
	%   a. which scenarios you plan to follow
	%   b. which activities from the glossary of activities you plan to execute in your ontology development project
	
	La Metodología NeOn para la construcción de redes de ontologías es una metodología basada en
	escenarios que se apoya en los aspectos de colaboración de desarrollo de ontologías y la
	reutilización, así como en la evolución dinámica de las redes de ontologías en entornos
	distribuidos
\cite{oeg-neon}.
	
	Como define esta metodología, hay un conjunto de nueve escenarios para la construcción de
	ontologías, haciendo hincapié en la reutilización de recursos ontológicos, la reingeniería y la
	fusión. Para este trabajo, en el cual se ha seguido la metodología NeOn se han seguido los
	siguientes escenarios:
	\begin{itemize}
		\item \textbf{Escenario 2: La reutilización y reingeniería de los recursos no ontológicos (NOR).}
		Se llevan a cabo procesos de reutilización NOR para decidir en base a los requisitos
definidos de la ontología que NORs se utilizarán para construir la red de la ontología.
Después de ello estos NORs se readaptarán mediante re-ingeniería.
		\item \textbf{Escenario 4: La reutilización y re-ingeniería de los recursos ontológicos.} 
		Se llevan a cabo procesos de reutilización de recursos y se reorganizan los recursos ontológicos.
		\item \textbf{Escenario 7: Reutilización de los patrones de diseño de ontologías (ODPs).}
		Los
desarrolladores de ontologías acceden a repositorios de reutilización ODPs.
		\item \textbf{Escenario 9: Localización de recursos ontológicos.}
		Los desarrolladores de ontologías
adaptan una ontología a otras lenguas y la cultura las comunidades, obteniendo así una
ontología multilingüe.
	\end{itemize}
	
	Otro punto importante dentro de la metodología NeOn es el glosario de actividades \cite{oeg-glossary}. Durante todo el proceso de construcción de
nuestra ontología se han realizado actividades incluidas en este glosario. A continuación se listan
algunas de las más importantes:
	
	\begin{itemize}
		\item \textbf{Ontology Requirements Specification.} Planteamiento de preguntas y la resolución de
estas para detectar las necesidades y requisitos de la ontología.
		\item \textbf{Ontology Aligning.} Buscamos entre los diferentes recursos ontológicos los elementos comunes para evaluar cual es más adecuada para nuestra ontología.
		\item \textbf{Ontology Annotation.} A los elementos de nuestra ontología les hemos asignado etiquetas
(labels) y comentarios para enriquecerla.
		\item \textbf{Ontology Comparison.} A la hora de buscar ontologías ya creadas sobre el mismo dominio
tuvimos que comparar y evaluar las diferencias entre ellas para elegir las mas adecuadas.
		\item \textbf{Ontology Conceptualization.} Durante el proceso de adquisición tuvimos que organizar
toda la información recopilada para crear un modelo conceptual que cubriese los
requisitos definidos.
		\item \textbf{Control.} Se siguió la planificación temporal prevista para completar las tareas de
desarrollo programadas.
		\item \textbf{Ontology Design Pattern Reuse.} Se han reusado patrones de diseño para nuestra
ontología, concretamente el patrón de Normas y el de Servicios.
		\item \textbf{Ontology Diagnosis.} Gracias al uso de herramientas como OOPS! pudimos evaluar los
fallos de la ontología y corregirlos durante el proceso de desarrollo.
		\item \textbf{Ontology Elicitation.} Se realizó una adquisición de estructuras conceptuales como el T-Box de documentos de expertos del dominio de las instalaciones deportivas.
		\item \textbf{Ontology Erichment.} Hemos extendido con más relaciones y clases algunas de las
ontologías que hemos reusado para completarlas para cubrir ciertos requisitos.
		\item \textbf{Ontology Implementation.} Una vez diseñado el modelo conceptual se transformó la ontologia a un modelo formal (código OWL) gracias a Protégé.
		\item \textbf{Ontology Localization.} Las entidades y relaciones creadas se etiquetaron tanto en español
como inglés.
		\item \textbf{Ontology Merging.} Se fusionaron clases y funciones de distintas ontologías para crear
una nueva.
		\item \textbf{Ontology Search.} Se realizó una búsqueda de ontologías o módulos de ontologías
candidatas para ser reutilizadas.
		\item \textbf{Ontology Module Extraction.} Se han extraído módulos concretos de ciertas ontologías.
		\item \textbf{Ontology Module Reuse.} Se han reutilizado los módulos extraídos de otras ontologías.
		\item \textbf{Non-Ontological Resource Reengineering.} Se han reestructurado contenidos no
ontológicos para adaptarlos a nuestra ontología.
		\item \textbf{Non-Ontological Resource Reuse.} Se han usado recursos no ontológicos para
transformarlos en partes de ontologías.
		\item \textbf{Scheduling.} Se ha realizado una planificación de las actividades anteriores para
completarlas en el orden correcto y así llevar un control de las mismas.
	\end{itemize}
	
	\section{Especificación de la ontología}
	% Ontology specification. You should include a complete ontology specification requirement document using the template explained during the lectures. The goal of the ontology should be clear enough. The document should include relevant competency questions and their answers.
	
	\subsection{Propósito}
	El objetivo de esta Ontología se enmarca en el proyecto Ciudades abiertas, que tiene como objetivo representar datos de ciudades en diversas areas. el objetivo de esta ontología sera modelar el conocimiento de Instalaciones Deportivas para que pueda ser utilizado como ontolodía dentro del proyecto.
	
	\subsection{Ámbito}
	La definición de Instalaciones Deportivas puede llegar a ser ambigua, dado que existen multiples instalaciones donde se realizan deporte con un fin muy distinto: un estadio de futbol y un parque de barras podrían ser ambos considerados instalaciones deportivas.
	
	Nuestra ontología se enmarca en la definición propuesta por el informe estadístico de 2022 publicado por el Ministerio de Educación, Cultura y Deporte, donde se proporcionan indicadores estadísticos para estimar las dimensiones y características de las infraestructuras en España. En el documento, se define instalación deportiva como: 'Instalaciones destinadas al
deporte que incluyen uno o varios espacios
deportivos donde puede desarrollarse la actividad
físico-deportiva.'.
	
	\subsection{Lenguaje de la Implementación}
	La ontología sera implementada en OWL utilizando .
	
	\subsection{Potenciales usuarios}
	Los usuarios identificados como posibles interesados en el uso de la ontología han sido los siguientes.
	\begin{enumerate}
		\item Ayuntamiento interesado en el proyecto ciudades abiertas que quiera aumentar la transparencia de los datos de sus instalaciones deportivas.
		\item Persona responsable de una instalación deportiva que quiera almacenar estadísticas de la misma y/o crear una página web de la instalación.
		\item Deportista que quiera conocer que instalación deportiva le es más conveniente.
		\item Ministerio u órgano del gobierno que quiera analizar estadísticas sobre el uso de las instalaciones deportivas en su país.
	\end{enumerate}

	\subsection{Potenciales casos de uso}
	Los posibles casos de uso identificados son los siguientes.
	\begin{enumerate}
		\item Conocer los subespacios forman una Instalación Deportiva concreta.
		\item Buscar servicios deportivos como clases o alquileres de espacios ofrecidos por la organziación que gestiona la Instalación Deportiva.
		\item Enumerar los deportes que pueden realizarse en cada Instalación Deportiva, y en que subespacio se realizan.
		\item Obtener estadísticas del uso de servicios ofrecidos en la Instalación Deportivapor cliente.
		\item Obtener información sobre los trabajadores de la organización que lleva la instalación deportiva.
	\end{enumerate}
	
	\subsection{Requisitos de la ontología}
	\subsubsection{Requisitos no funcionales}
	Los requisitos no funcionales de la ontología se refieren a las características y aspectos generales no directamente relacionados con el contenido de la ontología. En nuestro caso son los siguientes:
	\begin{itemize}
		\item NFR1: La ontología deberá soportar otros idiomas a parte de Español.
		\item NFR2: La documentación asociada a la ontología debe incluir fuentes fiables.
		\item NFR3: El formato utilizado para representar el conocimiento será OWL.
		\item NFR4: El sistema de nombrado en la ontología sera comprensible y coherente.
	\end{itemize}
	\subsubsection{Requisitos funcionales:  preguntas de competencia}
	Para las preguntas de competencia, hemos optado por un acercamiento \textit{Middle out}. Comenzamos realizando preguntas \textit{Bottom-Up}, sin mirar bases de datos ni contenido relacionado, simplemente buscando preguntas sencillas que podrían estar relacionadas con una instalación deprotiva. Cuando llegamos a un limite donde nuestras preguntas carecían de complejidad, entonces tuvimos un acercamiento \textit{Top-Down}, donde ya comenzamos a mirar información acerca de Instalaciones Deportivas, y pudimos desarrollar preguntas más complejas. El resultado puede verse en la tabla:
	\begin{figure}[H]
		\centering
		\includegraphics[width=0.9\textwidth]{include/preguntas_competencia.png}
		\caption{Preguntas de competencia}
	\end{figure}
	
	\subsection{Glosario de términos}
	
	Tras realizar las preguntas de competencia, hemos contado los sustantivos relevantes que se repetían, tanto en las preguntas para identificar clases y relaciones, como en las respuestas. 
	
	Cabe destacar que las preguntas se han contestado de manera muy genérica, sin concretar para una Instalación específica, de forma que los términos recogidos están un nivel por encima de ser instancias. Los resultados de algunos de los términos más relevanten se recogen en la siguiente figura: 

	\begin{figure}[H]
		\centering
		\includegraphics[width=0.9\textwidth]{include/terms.png}
		\caption{Glosario de términos}		
	\end{figure}
	
	\section{Planificación temporal de la ontología}
	% Ontology Schedule. You should identify the ontology life cycle model you use in your ontology development. Activities should come from the glossary of terms. Include a Gantt chart with the planned activities.
		
	El ciclo de vida utilizado en el diseño y desarrollo de esta ontología es el ciclo de vida incremental, pudiéndose considerarse como la primera iteración de un modelo ágil.
	
	Se decidió utilizar un ciclo de vida incremental debido a las características del proyecto, ya que la ontología no fue diseñada a partir de un set de datos proporcionados por el cliente, si no que se realizó el modelado a partir de las posibles necesidades de sus usarios. 
	
	La planificación temporal se puede dividir en varias secciones. Durante la planificación de la ontología se buscó el ámbito sobre el que realizar el trabajo, un tema sobre el que no existiesen ontologías ya diseñadas, y se definieron los requisitos planteando las preguntas de competencia. Se diseñó la ontología reutilizando los recursos útiles encontrados plasmándolos en el modelo conceptual. A continuación se creó la implementación del modelo y su evaluación. La redacción de este documento se puede considerar un ejercicio transversal.
	
	\begin{figure}[H]
		\centering
		\includegraphics[width=\textwidth]{include/gantt_ontologia.png}
		\caption{Diagrama de Gantt de la planificación temporal seguida.}
	\end{figure}
	
	\section{Búsqueda de ontologías de alto nivel}
	% Search for existing top level and domain ontologies that you plan to reuse when building your ontology. Explain if you need to reuse the ontology as a whole or if you need some modules or statements. Explain in which repositories you made the search, ontologies found, their relevance to your work and the criteria being used for selecting or withdrawing some of them.
	
	Una vez identificados el glosario de términos mediante la especificación, y creado un modelo donde se relacionan las diferentes entidades, el siguiente paso es la formalización de la ontología. Para ello, hemos realizado una búsqueda de ontologías de alto nivel, detectando aquellas que cubren de manera total o parcial secciones de nuestro modelo para poder reutilizarlas.
	
	Las primeras ontologías que hemos identificado para nuestra ontología, fueron mencionadas a lo largo del curso:
	\begin{itemize}
		\item \textbf{Org.} La ontología org ha sido utilizada para representar la entidad de la organización que lleva la instalación deportiva. No hemos utilizado la ontología completa, solo los módulos que nos hacían falta para representar la organización, su localización y sus miembros. Esta ontología es ampliamente utilizada: en LOV, podemos ver que 37 ontologías reutilizan contenido de org.
		\item \textbf{Foaf.} Para poder representar a todas las personas que pertenecen a nuestro modelo (director, entrenadores, monitores, deportistas...) hemos utilizado foaf. A efectos prácticos, nos ha servido con importar la ontología org, ya que esta contenía las entidades foaf que nos hacían falta
		\item \textbf{Odrl.} La ontología ODRL permite modelar reglas y políticas. Se ha utilizado para representar las reglas de vestimenta en los espacios deportivos. Para ello, se ha utilizado únicamente parte de la ontología, en concreto las entidades que son necesarias para generar el patrón Norm, que describiremos más adelante.
		\item \textbf{Geo.} La ontología Geo se ha utilizado para poder expresar la geolocalización de la instalación mediante la entidad GeoPoint. Esta ontología no tiene propiedades, así que hemos tenido que crear una propiedad que relacione la instalación con el GeoPoint. 
		\item \textbf{Vcard.} Consiste en una ontología utilizada para representar la información que contiene una tarjeta de visita. En nuestro caso, hemos utilizado únicamente las entidades referentes a la dirección (Address) y nombre (OrganizationName).
		
	\end{itemize}
	Una vez identificadas estas ontologías de alto nivel, utilizamos el buscador de LOV (Linked Open Vocabularies) para buscar otras posibles ontologías que pudiéramos reutilizar. En concreto encontramos las siguientes:
	
	\begin{itemize}
		\item \textbf{Time.} La ontología time permite expresar tanto fechas como intervalos y duración de eventos. En nuestro caso, para expresar la fecha y la duración de los servicios hemos utilizado dos entidades diferentes.
		\item \textbf{Sport.} Esta ontología estaba disponible en el buscador, pero las URIs no llevaban a ninguna ontología, por lo que no pudimos reutilizarla. 
	\end{itemize}
	A la hora de importar todas estas ontologías en , utilizamos LOV como método de búsqueda de las URIs, lo que facilitó la tarea. 
	
	\section{Recursos no ontológicos}
	%Search for non ontological resources and other terminologies that could be transformed into ontologies. Keep track of the URLs where you found them. For the selected resources, do not forget to justify why you have selected them. Check if you have Access rights for using them within your hands-on assigment.
	
	Una de las fuentes de información utilizadas para generar la ontología propuesta en este documento es el siguiente informe \cite{pdf-culturaydeporte}. A partir de este recurso no ontológico, mediante el uso de un T-Box se pudo realizar una modelización de la información descrita que representa la estructura taxonómica del documento en esta ontología de instalaciones deportivas. 
	
	El documento \cite{pdf-culturaydeporte} del Ministerio de Cultura y Deporte del Gobierno de España, contiene los  principales resultados del informe de explotación estadística del censo de instalaciones deportivas de 2005, así como las definiciones de las diferentes clases de instalaciones deportivas consideradas durante el censo.
	
	Al plasmar el conocimiento presente en este documento en el modelo de la ontología mediante un T-Box, se decidió mantener las clases intermedias presentes en el documento como parte de la ontología pese a que serán probablemente rara vez utilizadas para mantener la estructura del documento mencionado y facilitar la reutilización de este. 
	
	\begin{figure}[H]
		\centering
		\includegraphics[width=0.7\textwidth]{include/tbox.jpg}
		\caption{Sección correspondiente a los recursos no ontológicos en la ontología.}
	\end{figure}
	
	Es importante señalar que la estructura taxonómica del documento descrito exige desambiguar entre las relaciones "sub-clase de" (\textit{sc} en el diagrama) y "parte de" (\textit{PoF} en el diagrama).
	
	\section{Patrones en la ontología}
	% Search for some ontology design patterns in the ontology design pattern portal that could be reused in your development.
	
	A la hora de diseñar nuestra ontología, de entre los patrones de diseño ya existentes vimos que
algunos de ellos eran aplicables a nuestro caso de instalaciones deportivas. Concretamente, los
patrones que hemos implementado han sido el de “Servicios ofrecidos” y el de “Normas”.
	
	Para el diseño de los servicios ofrecidos por la organización en una instalación deportiva, se
reutilizó el patrón de una relación N-aria según lo visto en clase. Este patrón se utiliza para
representar una relación N-aria en el que todos los elementos tienen la misma importancia. Para
representar esta relación N-aria se crea una clase, en nuestra ontología la clase ServicioOfrecido,
a la que asociar todos los atributos de la relación. Este patrón se utilizará para representar los
servicios que una organización ofrece que se realizaran en una instalación deportiva en un horario
determinado y por un precio indicado.
	
	\begin{figure}[H]
		\centering
		\includegraphics[width=0.65\textwidth]{include/patron_servicios.png}
		\caption{Esquema del patrón de servicios.}
	\end{figure}

	El patrón de normas está destinado a la definición de una serie de reglas y normas que aplican a
la organización. El patrón de reglas se expresa mediante ODRL (Open Digital Rights Language),
y se definen mediante la estructura:	“Una \textit{acción} es \textit{permitida/obligatoria/prohibida} de ser realizada por una \textit{parte} sobre el \textit{activo},
siempre que se mantengan las \textit{restricciones}”.
	
	\begin{figure}[H]
		\centering
		\includegraphics[width=0.5\textwidth]{include/patron_reglas.png}
		\caption{Esquema del patrón de servicios.}
	\end{figure}
	
	En la sección de modelado conceptual de la ontología se entrará más en detalle sobre las clases y
relaciones de estos patrones.
	
	\section{Modelo conceptual}
	% Build a conceptual model that integrates outcomes from the previous sections (c,d,e). This is the most important part of the work you are doing. Try to use:
	%  	i. Top level ontologies and other well-known ontologies. Classify in the pyramid of ontologies (figure use vs reuse) each of the ontologies that you reuse.
	%   ii. Transform each non-ontology resource into an ontology by using the T-box, A-Box or Population.
	%   iii. Select some Ontology Design Patterns (events, sequence, etc.)
	%   iv. Build the conceptual model of your ontology by integrating the above sources. The conceptual model should have at least 40 concepts, several subclass-of relations, disjoint, part-of (if needed), and ad-hoc relations.
	
	Tras el proceso de búsqueda de recursos ontológicos y no ontológicos para su reutilización
como se ha explicado en los apartados anteriores así como la selección de patrones de diseño de
ontologías, se desarrolló el modelo conceptual de la ontología. Esta es una de las fases más
importantes en la construcción de ontologías, y en este apartado se describirán las
clases y relaciones que se han implementado y que en los siguientes apartados se formalizarán a
código OWL. Debido a la amplitud del modelo y para un mejor entendimiento se procederá a
explicar los distintos módulos del modelo por separado.
	
	Encontramos cuatro clases principales que serán el núcleo de nuestra ontología.
Estas clases son \textbf{Organización}, \textbf{Instalación Deportiva}, \textbf{Servicio Ofrecido} y \textbf{Regla}.
	
	\subsection{Organización}
	
	En lo que a la organización respecta, reutilizando clases y relaciones de la ontología \textbf{org} hemos
creado el modelo que se puede observar en la figura a continuación. La forma de interpretar esta parte del
modelo es: “Una organización encargada de gestionar instalaciones deportivas establece acuerdos
de membresía con sus trabajadores para que estos tengan un determinado rol dentro de la
organización”. Analizando el modelo con sus clases y relaciones tenemos:
	
	\begin{itemize}
		\item Una organización es SubClassOf de \textit{org:Organization}. Organización tiene como
propiedades el Nombre, Email y Número de teléfono, que son literales.
		\item Una organización es la que gestiona una o más instalaciones deportivas, por lo que la
clase Instalación Deportiva se relaciona con Organización mediante la relación
\textit{managedBy}.
		\item Una organización establece unos acuerdos de membresía con sus trabajadores para un
determinado rol (tal y como está definido en la ontología org) por lo que tenemos la clase
\textit{org:Membership} que se relaciona con organización mediante la relación
\textit{org:organization}, se relaciona con las clases de los trabajadores mediante la relación
\textit{org:memberOf} y se relaciona con org:Role mediante la relación org:role. Al diseñar esta
sección hemos querido contemplar la posibilidad de que un trabajador de la organización
tenga asignada la instalación o las instalaciones en las que trabaja por lo que hemos unido
\textit{org:Membership} con Instalación Deportiva mediante la relación \textit{establecidaEn}.
		\item Para determinar los tipos de trabajadores que hay en una organización hemos decidido
crear una clase para cada uno de los tipos. Debido a la gran cantidad de clases de
trabajadores que existen para su simplificación en nuestro modelo hemos ejemplificado
las clases Monitor, Limpiador, y Entrenador, todas ellas \textit{SubClassOf} de \textit{foaf:Person}.
	\end{itemize}

	\begin{figure}[H]
		\centering
		\includegraphics[width=\textwidth]{include/org.jpg}
		\caption{Esquema de las principales clases relacionadas con \textbf{Organización}.}
	\end{figure}
	
	\subsection{Instalación Deportiva}
	
	Para la construcción de la taxonomía de las instalaciones deportivas se ha hecho uso del recurso
no ontológico explicado en apartados anteriores, y mediante un T-Box se ha modelado a nuestra
ontología.
	
	\begin{itemize}
		\item Una Instalación Deportiva puede estar compuesta por diferentes partes (\textit{partOf}):
Espacio Deportivo (que puede ser al aire libre o cerrado), Espacio Complementario,
Espacio Auxiliar. Estas partes son disjuntas unas de otras. Un Espacio
Complementario va asociado a un Espacio Deportivo.
		\item Cada una de estas partes tiene su propia taxonomía tal y como se define en el recurso no
	ontológico y se muestra las diferentes subclases y las clases que son disjuntas (recuadros
rosas) en el esquema de la Figura X2.
		\item La clase Instalación Deportiva cuenta con propiedades como número de teléfono e email que
serán literales y no clases.
		\item Una instalación deportiva se geolocaliza en un \textit{geo:Point} mediante la relación
\textit{hasGeopoint}. A su vez expresamos su localización mediante una dirección, para ello
relacionamos Instalación Deportiva con la clase \textit{vcard:Address} mediante la relación
\textit{vcard:hasAddress}.
		\item La clase Deporte se relaciona con un Espacio Deportivo mediante la relación
\textit{esPracticadoEn}.
		\item La clase Cliente es \textit{subClassOf} \textit{foaf:Person}, y Cliente practica uno o varios Deportes.
En el siguiente modulo también se verá que Cliente consume ServicioOfrecido
		\item Entrenador y Monitor (definidos en el apartado de organización) tienen una relación
	con Deporte que es \textit{enseña}.
	\end{itemize}

	\begin{figure}[H]
		\centering
		\includegraphics[width=\textwidth]{include/tbox.jpg}
		\caption{Esquema de las principales clases relacionadas con \textbf{Instalación Deportiva}.}
	\end{figure}
		
	\subsection{Servicio Ofrecido}
	
	Esta clase ServicioOfrecido tendrá asociadas varias clases mediante diferente relaciones:
	
	\begin{itemize}
		\item Servicio, que tiene dos subclases disjuntas Individual y Bono de servicios. De estas dos
	clases habría más subclases, para ejemplificar hemos incluido el alquiler de una pista de
	pádel o un bono de 10 clases de zumba
		\item Instalación Deportiva en la que se ofrece dicho servicio. Por ejemplo, una pista de pádel,
o la sala de musculación.
		\item time:DurationDescription y time:GeneralDateTimeDescription, que indicaran la
duración y/o la fecha en la que se ofrece el servicio. Por ejemplo, jueves 3 de noviembre	del 2022 de 18:00 a 19:00, o Todos los lunes y miércoles de febrero de 15:00 a 16:00.
		\item PriceSpecification, teniendo asociada como propiedades una cantidad y una moneda de
pago. Por ejemplo 59,99 euros.
	\end{itemize}
	
	\begin{figure}[H]
		\centering
		\includegraphics[width=\textwidth]{include/pattern.jpg}
		\caption{Esquema de las principales clases relacionadas con \textbf{Servicio Ofrecido}.}
	\end{figure}
	
	\subsection{Regla}
	
	Al incluir este patrón de normas, hemos creado una única regla para ejemplificar las normas de
	una organización (en la realidad habría muchas de ellas, pero para simplificar el modelo para este
	trabajo se ha trabajado solo con una). Las reglas constan de la clase Rule que define si se trata de
	una prohibición, una obligación o permiso, y de ella salen varias relaciones a una serie de clases
	que serán subclass of de las siguientes entidades de la ontología ODRL:
	
	\begin{itemize}
		\item \textbf{odrl:Constraint}. Para cada norma hay una subclass de \textit{odrl:Constraint} y puede ser un
lugar físico (piscina, pista de tenis, gimnasio) o un marco temporal (los sábados, el día 1
de cada mes, todos los días de 9:00 a 10:00) en el que se aplica la norma.
		\item \textbf{odrl:Asset}. Para cada norma tendremos una clase \textit{subclass of} \textit{odrl:Asset} que establece el
activo o recurso que es sujeto de la norma. Este activo puede ser cualquier forma de
recurso identificable, como datos/información, servicios o elementos físicos. En nuestra
ontología hemos puesto el ejemplo de gorro de baño.
		\item \textbf{odrl:Action}. Para cada norma tendremos una clase \textit{subclass of} \textit{odrl:Action} que indicará
la acción realizada sobre el activo. En nuestro modelo encontramos la clase Llevar puesto
para el caso de gorro de baño.
		\item \textbf{odrl:Party} Tendremos una \textit{subclass} de esta clase de ODRL para cada regla, y definirá a
que parte de las personas se le aplica dicha regla. Puede ser por ejemplo una regla que
aplique a todos los usuarios, o una regla que aplique solo a trabajadores.
	\end{itemize}
	
	\begin{figure}[H]
		\centering
		\includegraphics[width=\textwidth]{include/reglas.jpg}
		\caption{Esquema de las principales clases relacionadas con \textbf{Regla}.}
	\end{figure}
	
	\section{Implementación de la ontología con OWL}
	% Implement the ontology in an ontology development tool, or other ontology editor, using OWL as ontology language
	
	Una vez identificados el glosario de términos mediante la especificación, y hemos conceptualizado un modelo donde se relacionan las diferentes entidades, el siguiente paso, la formalización de la ontología. Para ello, hemos realizado una búsqueda de ontologías de alto nivel, de forma que si alguna de ellas cubre toda o parte de alguna de nuestras entidades, podamos \textbf{reutilizarla.
	}.\\
	Las primeras ontologías que hemos identificado para nuestra ontología, fueron mencionadas a lo largo del curso:
	\begin{itemize}
		\item \textbf{Org:}La ontología org ha sido utilizada para representar la entidad de la organización que lleva la instalación deportiva. No hemos utilizado la ontología completa, solo los módulos que nos hacían falta para representar la organización, su localización y sus miembros. Esta ontología es ampliamente utilizada: en LOV, podemos ver que 37 ontologías reutilizan contenido de org.
		\item \textbf{Foaf: }Para poder representar a todas las personas que pertenecen a nuestro modelo (director, entrenadores, monitores, deportistas...) hemos utilizado foaf. A efectos prácticos, nos ha servido con importar la ontología org, ya que esta contenía las entidades foaf que nos hacían falta
		\item \textbf{Odrl:} La ontología ODRL permite modelar reglas y políticas. Se ha utilizado para representar las reglas de vestimenta en los espacios deportivos. Para ello, se ha utilizado únicamente parte de la ontología, en concreto las entidades que son necesarias para generar el patrón Norm, que describiremos más adelante.
		\item \textbf{Geo:} La ontología Geo se ha utilizado para poder expresar la geolocalización de la instalación mediante la entidad GeoPoint. Esta ontología no tiene propiedades, así que hemos tenido que crear una propiedad que relacione la instalación con el GeoPoint. 
		\item \textbf{Vcard: }Consiste en una ontología utilizada para representar la información que contiene una tarjeta de visita. En nuestro caso, hemos utilizado únicamente las entidades referentes a la dirección (Address) y nombre (OrganizationName).
		
	\end{itemize}
	Una vez identificadas estas ontologías de alto nivel, utilizamos el buscador de LOV (Linked Open Vocabularies) para buscar otras posibles ontologías que pudieramos reutilizar. En concreto encontramos las siguientes:
	
	\begin{itemize}
		\item \textbf{Time:} La ontología time permite expresar tanto fechas como intervalos y duración de eventos. En nuestro caso, para expresar la fecha y la duración de los servicios hemos utilizado dos entidades diferentes.
		\item \textbf{Sport:} Esta ontología estaba disponible en el buscador, pero las URIs no llevaban a ninguna ontología, por lo que no pudimos reutilizarla. 
	\end{itemize}
	A la hora de importar todas estas ontologías en , utilizamos LOV como métode de búsqueda de las URIs, lo que facilito la tarea. 
	Como se ha especificado en los requisitos, la onotlogía estará implementada en OWL. Para eso, hemos utilziado Protégé, un editor de ontologías open source. \\
	
	El primer paso que hemos hecho ha sido incluir todas las clases de la ontología que hemos creado mediante el T-Box. Después, hemos reutilizado de las ontologías nombradas en el apartado 5 todas las clases que nos hacían falta. Para ello, hemos importado las ontologías con su URI y movido los axiomas necesarios a nuestra ontología.
	
	En general, la metodología para importar las ontologías ha sido idéntica para todas: primero hemos importando la ontología al completo con su uri, moviendo los axiomas que necesitábamos, y después hemos eliminando el resto. La única excepción ha sido la ontología Time. 
	
	Hemos importado la ontología Time de forma completa, ya que utilizábamos la totalidad de sus entidades. Al importar la ontología, nos hemos dado cuenta de que muchas de las propiedades estaban incompletas: les faltaba el rango o el dominio. Hemos optado por mantenerla como estaba, ya que muchos de estos problemas se debían a términos deprecados, y no hemos encontrado una versión más reciente de la ontología.
	
	Cabe destacar que todas las clases tienen etiquetas estilo \textit{label} en español e inglés.
	
	Una vez importadas las clases necesarias, hemos definido aquellas que nos hacían falta para completar nuestro modelo, que no se encontraban ni en el T-Box ni en las ontologías de alto nivel, como puede ser por ejemplo la clase Monitor, Entrenador, Pista de Baloncesto... \\
		
	Este mismo procedimiento lo hemos realizado para los Object Properties. En este caso, a demás de importar los necesarios de las diferentes ontologías y crear los que necesitabamos, hemos definido las propiedades de cada propiedad. \\
		
	Hemos tenido cierta dificultad a la hora de crear la propiedad Part Of, ya que tiene varios dominios para un único rango (tanto los espacios deportivos, como los complementarios y auxiliares son part of de una instalación deportiva). Para evitar interferencias, hemos incluido los diferentes dominios con la condición OR. 
	
	De forma similar hemos añadido los Data Properties. Estos en lugar de apuntar a una entidad, apuntan a un literal, por lo que hemos tenido que especificar los tipos de datos para literales como la divisa, el precio... Un caso particular ha sido la propiedad hasCeiling (tiene techo), para definir los Espacios Deportivos. En este caso el literal era un booleano.
\\
	
	Finalmente, hemos completado las labels y comentarios de aquellas entidades que no han sido directamente importadas, asegurandonos de que estaban disponibles tanto en inglés como español. 
	
	En el anexo se puede encontrar una imagen de la jerarquía de las clases, los Object Properties y los Data Properties presentes en la ontología.
	
	Una vez finalizada la implementación de la ontología, la hemos exportado como RDF para poder evaluarla con OOPS!
	
	\section{Evaluación de la ontología con OOPS!}
	% Evaluate the ontology with OOPS! and include in your report the pitfalls found. It is highly advisable to combine this evaluation with other ontology evaluation techniques.
	% Improve your ontology (conceptual model and implementation) taking into account the suggestions given by OOPS!. Iterate in these steps until the ontology pass most of the OOPS! recommedations.
	La ontología diseñada se ha evaluado utilizando la herrramienta OntOlogy Pitfall Scanner! (OOPS!)\cite{oops}, que identifica errores en la ontología dividiéndolos en críticos, importantes y menores, de acuerdo a una batería errores comunes.
	
	\subsection{Mejoras implementadas tras la sugerencia de OOPS!}
	
	Tras la primera iteración de la evaluación con OOPS!, la herramienta destacó los siguientes problemas críticos:
	
	\begin{itemize}
		\item \textbf{P19:} Propiedades con múltiples rangos o dominios.
		
		Una de las relaciones definidas tenía más de un dominio, habiendo escrito en  accidentalmente \textit{and} en lugar de \textit{or} en el dominio.
		
		Los dominios de las relaciones \textit{time:hasDateTimeDescription} y \textit{time:hasTemporalDuration} fueron modificados añadiendo una entrada en sus dominios y eliminando la entrada de la ontología time. Esta modificación no fue aplicada por Protégé, y hubo que modificar la entrada del dominio impuesta por importar la ontología time para que el cambio quedase registrado.
		
		\item \textbf{P29:} Relaciones transitivas mal definidas.
		
		Las relaciones \textit{partOf}, \textit{hasService}, \textit{managed\_by} y \textit{constraint} estaban definidas como transitivas, teniendo rangos y dominios distintos. Por definición una propiedad transitiva debe tener el mismo rango y dominio.
		
		En el caso de \textit{partOf}, una relación que generalmente es transitiva, decidimos eliminar esta propiedad, ya que para nuestro modelo aparece únicamente entre Instalación Deportiva y los espacios deportivos, complementarios y auxiliares, que nunca serán unos parte de otros.
		
		Las otras relaciones se encontraban en esta clasificación por un error en la transcripción a Protégé.
	\end{itemize}
	
	Las propiedades \textit{vcard:hasEmail} y \textit{vcard:hasTelephone} aparecen como propiedades sin rango (P11), marcado como error importante, por ser su rango una clase deprecada de la ontología vcard. Pese a estar deprecada, esto no es un error, por lo que no se solucionó el problema.
	
	Gracias a los errores destacados por OOPS! se pudieron solucionar todos los errores críticos y varios errores clasificados como importantes. Se decidió no solucionar los errores no críticos de las ontologías importadas debido a las restricciones temporales sobre la entrega del trabajo y a la prioridad que se impuso sobre la reutilización de esta ontología.
	
	\subsection{Resultados finales de la evaluación}
	
	Tras aplicar las correcciones sobre los errores críticos indicados por OOPS! la última versión de la ontología tiene $11$ errores importantes, $63$ errores menores y $16$ sugerencias, mientras que en su primera iteración tenía $8$ errores críticos, $12$ errores importantes y $70$ errores menores, así como $16$ sugerencias.
	
	Cabe destacar que aproximadamente el 40\% de los errores indicados originalmente por OOPS! estaban relacionados con la ontología importada time.
	
	Los resultados completos de ambas evaluaciones se pueden encontrar en el Anexo.
	
	\section{Documentación de la ontología}
	% Document the ontology with Widoco and use Ontoology if required
	
	
		
	\cite{widoco}
	
	%\includepdf[pages=-]{include/documentation-widoco.pdf}
	
	\section{Conclusiones}
	
	
\newpage
	\section*{Bibliografía}
	\addcontentsline{toc}{section}{Bibliografía}
	\bibliography{include/references}
	\bibliographystyle{IEEEtran}
	
	\newpage
	\section*{Anexos}
	\addcontentsline{toc}{section}{Anexos}
	
	\begin{figure}[H]
		\centering
		\includegraphics[height=\textheight]{include/classes.png}
		\caption{Jerarquía de clases de la ontología.}
	\end{figure}

	\begin{figure}[H]
		\centering
		\includegraphics[width=\textwidth]{include/object.png}
		\caption{Object Properties presentes en la ontología.}
	\end{figure}

	\begin{figure}[H]
		\centering
		\includegraphics[width=\textwidth]{include/data.png}
		\caption{Data Properties presentes en la ontología.}
	\end{figure}
	
	
	\begin{figure}[H]
		\begin{subfigure}{.5\textwidth}
			\centering
			\includegraphics[height=\textheight]{include/eval_inicial_oops.png}
			\caption{Resultados iniciales de la evaluación con OOPS! }
		\end{subfigure}
		\begin{subfigure}{.5\textwidth}
			\centering
			\includegraphics[height=\textheight]{include/eval_final_oops.png}
			\caption{Resultados finales de la evaluación con OOPS!}
		\end{subfigure}
		\caption{Diferentes resultados obtenidos de la evaluación con OOPS!}
	\end{figure}
	


\end{document}